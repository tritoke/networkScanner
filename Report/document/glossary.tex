% glossary entries
  \newglossaryentry{bbox}{%
    name={black box},
    description={%
      Looking at something from an outsider's perspective
      knowing nothing about how it works internally.
    }
  }

  \newglossaryentry{upow}{%
    name={upowerd},
    description={%
      Manages the power supplied to the system: charging, battery usage etc\ldots
    }
  }

  \newglossaryentry{sysd}{%
    name={systemd},
    description={%
      A daemon for controlling what is run when the system starts.
    }
  }

  \newglossaryentry{dbus}{%
    name={dbus-daemon},
    description={%
      A daemon which enable a common interface for inter-process communication.
    }
  }

  \newglossaryentry{proc}{%
    name={process},
    description={%
      When running programs on a computer a new ``process'' is started with each new program
      programs can also spawn their own ``child'' processes to do tasks for them while they do
      something else, this enables one program to do more than one thing at once.
    }
  }

  \newglossaryentry{ipaddr}{%
    name={IP address},
    plural={IP addresses},
    description={%
      Every computer on a network has a unique IP address assigned to them,
      which is used to identify where exactly message sent by computers are
      meant to go.
    }
  }

  \newglossaryentry{daemon}{%
    name={daemon},
    plural={daemons},
    description={%
      A process that runs forever in the background to facilitate other programs.
    }
  }

  \newglossaryentry{port}{%
    name={port},
    plural={ports},
    description={%
      Computers have ``ports'' for each protocol which can be connected to separately,
      this makes up part of a ``socket'' connection.
    }
  }

  \newglossaryentry{kernel}{%
    name={kernel},
    description={%
      The kernel is the foundation of an operating system and it serves as the main
      interface between the software running on the system and the underlying hardware
      it performs task such as processor scheduling and managing input/output operations.
    }
  }

  \newglossaryentry{driver}{%
    name={driver},
    plural={drivers},
    description={%
      A tiny software module which is loaded into the kernel when the
      computer boots up, They mainly interface with hardware and are
      often very specific for each piece of hardware.
    }
  }

  \newglossaryentry{server}{%
    name={server},
    plural={servers},
    description={%
      A server is any computer which it's purpose is to provide resources
      to others, either humans or other computers for purposes from
      hosting website or just as a resource of large computational power.
    }
  }

  \newglossaryentry{banner}{%
    name={banner},
    plural={banners},
    description={%
      A short piece of text which a service with send to identify itself
      when it receives a connection request. Often contains information
      such as version number etc\ldots
    }
  }

  \newglossaryentry{port knocking}{%
    name={port knocking},
    description={%
      Port knocking is where packets must be sent to a sequence of ports
      before access to the desired port is granted.
    }
  }

  \newglossaryentry{half open}{%
    name={half open scanning},
    description={%
      Half open scanning is where no full connection to the host is made,
      only one to solicit a response and then once that response is received
      no further packets are sent, leaving the connection ``half open''.
    }
  }

  \newglossaryentry{csum}{%
    name={checksum},
    plural={checksums},
    description={%
      A checksum is a value calculated from a mathematical algorithm which
      is sent with the packet to its destination to allow the recipient
      to check whether the packet was corrupted on the way.
    }
  }

  \newglossaryentry{pkt}{%
    name={packet},
    plural={packets},
    description={%
      Packets are simply a list of bytes which contains packed values
      such as to and from address and they are the basis for almost all
      inter-computer communications.
    }
  }

  \newglossaryentry{probe}{%
    name={probe},
    plural={probes},
    description={%
      A probe is a packet which contains a very specific piece of data
      which is targeted such that it will elicit a response from a target
      such as a ``GET'' request to a webserver would hopefully return some
      information about what is running on the webserver.
    }
  }

  \newglossaryentry{subnet}{%
    name={subnet},
    plural={subnets},
    description={%
      A subnet is simply the sub-network of every possible IP address that
      will be used for communication on a particular network.
    }
  }
  
  \newglossaryentry{service}{%
    name={service},
    plural={services},
    description={%
      A service is something running on a machine that offers a service
      to either other programs on the computer or to people on the internet.
    }
  }

  \newglossaryentry{header}{%
    name={header},
    plural={headers},
    description={%
      A header is the first few bytes at the start of a packet often
      consisting of information on where to send the packet next, can
      also contain information though.
    }
  }

% acronyms
  \newacronym{ids}{IDS}{Intrusion Detection System}

  \newacronym{dhcp}{DHCP}{Dynamic Host Configuration Protocol}

  \newacronym{dhcpcd}{DHCPCD}{Dynamic Host Configuration Protocol Client Daemon}

  \newacronym{nic}{NIC}{Network Interface Card}

  \newacronym{tcp}{TCP}{Transmission Control Protocol}

  \newacronym{udp}{UDP}{User Datagram Protocol}

  \newacronym{icmp}{ICMP}{Internet Control Message Protocol}

  \newacronym{cidr}{CIDR}{Classless Inter-Domain Routing}

  \newacronym{sctp}{SCTP}{Stream Control Transmission Protocol}

  \newacronym{ftp}{FTP}{File Transfer Protocol}

  \newacronym{oop}{OOP}{Object Oriented Programming}

  \newacronym{mac}{MAC}{Media Access Control}

  \newacronym{cpe}{CPE}{Common Platform Enumeration}

  \newacronym{osi}{OSI model}{Open Systems Interconnection model}

  \newacronym{api}{API}{Applications Programming Interface}

  \newacronym{http}{HTTP}{HyperText Transfer Protocol}

  \newacronym{https}{HTTPS}{HyperText Transfer Protocol Secure} 

  \newacronym{php}{PHP}{PHP Hypertext Processor}

  \newacronym{html}{HTML}{HyperText Markup Language}

  \newacronym{xml}{XML}{eXtensible Markup Language}

  \newacronym{dns}{DNS}{Domain Name System}

  \newacronym{ip}{IP}{Internet Protocol}

  \newacronym{pcap}{PCAP}{Packet CAPture}

  \newacronym{arp}{ARP}{Address Resolution Protocol}

  \newacronym{ssh}{SSH}{Secure SHell}
  
  \newacronym{rdp}{RDP}{Remote Desktop Protocol}
